\documentclass[10pt,leqno]{article}
\usepackage[polish]{babel}
%\usepackage[OT4]{polski}
\usepackage[utf8]{inputenc}
\usepackage[T1]{fontenc}
\frenchspacing
%\usepackage{indentfirst}
\usepackage{listings}
\usepackage{framed}
\usepackage[textwidth=16cm,textheight=24cm]{geometry}
\usepackage{graphicx}
%\graphicspath{{img\\}}

\setlength\fboxsep{0pt}
\setlength\fboxrule{0.5pt}

\newcommand{\myimage}[3]{
  \begin{figure}[h!]
    \centering
      \includegraphics[scale=#1]{#2}
  \caption{#3}
  \end{figure}
}

\usepackage{float}
\floatstyle{boxed} 
\restylefloat{figure}

\title{\LARGE Licencjacki projekt programistyczny 2011 \\ 
       \ \\
       Program do wykonywania obliczeń statystycznych związanych z grą go \\ 
       \ \\
       Podręcznik użytkownika }

% TODO autor wieksza czcionka

\author{Wojciech Jedynak}
\date{Wrocław, \today}

% \newtheorem{rys}{Rysunek}

\begin{document}

\maketitle 

\thispagestyle{empty}
\tableofcontents

\newpage

\section{Wprowadzenie}

\subsection{Cel dokumentacji}
Zadaniem niniejszego podręcznika jest opisanie wszystkich aspektów dotyczących użytkowania programu GoStat od instalacji poprzez konfigurację
i eksploatację do jego deinstalacji.

\section{Wymagania}
Program najlepiej działa pod systemami linuksowymi. Instalacja dla systemów z rodziny Windows jest możliwa, ale wymaga większej liczby kroków. 

By używać programu konieczne jest posiadanie około 20 Mb miejsca na dysku oraz przeglądarki internetowej, która obsługuje JavaScript. 
W razie konieczności kompilacji programu należy posiadać połączenie z Internetem oraz udostępnić około 500 Mb dla kompilatora GHC (o ile nie 
został on już wcześniej zainstalowany).

\newpage

\section{Instalacja i usuwanie programu}
Program jest dystrybuowany w dwu wersjach: w postaci binarnej oraz jako kod źródłowy.

\subsection{Postać binarna}
Należy: 
1) pobrać plik <GoStat-binary-os.tar.gz> (gdzie os to 'linux' bądź 'windows')
2) rozpakować archiwum do wybranego katalogu
3) przejść do ów folderu
4) uruchomić (najlepiej z poziomu terminala) plik GoStat (GoStat.exe w przypadku Windows)

W celu usunięcia programu wystarczy usunąć wymieniony wcześniej katalog oraz ew. wszystkie utworzone za jego pomocą bazy danych (pliki *.db).

\subsection{Kod źródłowy}
Należy:
1) rozpakować plik <GoStat-source.tar.gz>
2) przejść do katalogu GoStat
3) z poziomu terminala wydać polecenie <cabal install>
4) poczekać aż narzędzie cabal-install pobierze i zainstaluje wszystkie brakujące pakiety modułów Haskellowych (może to zająć do kilkunastu minut).

Jeśli instalacja przebiegnie pomyślnie program będzie dostępny po wydaniu polecenia <GoStat> (<GoStat.exe> dla Windows).

W celu usunięcia programu należy udać się do katalogu .cabal (domyślnie znajduje się on w katalogu domowym użytkownika) i usunuąć wszystkie
podfoldery, które w nazwie mają frazę <GoStat>. Ewentualnie skasować należy także wszystkie utworzone za pomocą programu bazy danych (pliki *.db).

\newpage

\section{Uruchamianie i zamykanie programu}

\subsection{Rozpoczęcie pracy}
Aby rozpocząć pracę z programem należy wydać polecenie \\
<GoStat> \\ 
\\
Gdy program odpowie \\
<Listening on port 8000...> \\
należy otworzyć przeglądarkę internetową i wskazać adres \\
<http://localhost:8000> \\
\\
Powinna wówczas załadować się strona startowa:

\myimage{0.4}{start.png}{Strona startowa}

\subsection{Kończenie pracy}
Aby zakończyć działanie programu należy zamknąć okno terminala bądź wysłać sygnał zakończenia (Control-C pod Linuks, Control-Z pod Windows).

UWAGA. Po wykonaniu tej czynności nie będzie można używać interfejsu WWW -- otrzymamy komunikat ``serwer nie odpowiada''. Aby przywrócić działanie programu
wystarczy go ponownie uruchomić.


\newpage

\section{Konfiguracja programu}
Aby skonfigurować program, należy go: uruchomić (patrz rozdział: <Rozpoczęcie pracy>) i kliknąć odnośnik <Konfiguracja> widoczny na <Rysunku 1>.

Ukaże się wówczas następujący formularz:

\myimage{0.4}{formularz.png}{Formularz konfiguracyjny}

Składa się on z dwóch pól tekstowych. 

W pierwszym z nich należy podać lokalizację pliku bazy danych o rozszerzeniu .db, którego GoStat użyje do zebrania informacji o podanych zapisach gier go.
Plik nie musi istnieć fizycznie na dysku -- zostanie on utworzony w razie potrzeby -- ale jeśli podany zostanie katalog, to wymagane jest, aby został on
wcześniej utworzony.

W drugim polu należy podać pełne (tj. bezwględne) scieżki do katalogów zawierających pliki .sgf z zapisami gier, które chcemy analizować przy pomocy
programu GoStat. W każdym wierszy pola tekstowego można podać osobną ścieżkę. \\
\textbf{Ważne:} nie trzeba podawać każdego katalogu osobno, gdyż program szuka gier we \textbf{wszystkich podkatalogach} podanych folderów.

Aby zapisać zmiany w konfiguracji, należy kliknąć <Zapisz ustawienia>. Wówczas automatycznie wrócimy do strony startowej (<Rysunek 1>). 
Aby wykonane zmiany były widoczne w przeglądarce ruchów, należy następnie wybrać <Przebudowa bazy danych>. Opcja ta jest opisana poniżej.

\newpage

\section{Utworzenie i wypełnienie bazy danych}
Aby utworzyć (przebudować) bazę danych i wypełnić ją danymi, należy: 
  uruchomić program (patrz rozdział: <Rozpoczęcie pracy>)
  kliknąć odnośnik <Przebudowa bazy danych> widoczny na <Rysunku 1>.

Ponieważ przebudowa istniejącej bazy danych zaczyna się się od skasowania poprzedniej tabeli, użytkownik zostanie
poproszony o potwierdzenie swego zamiaru:

\myimage{1.0}{potwierdzenie.png}{Potwierdzenie utworzenia bazy danych}

Po kliknięciu <OK> należy poczekać aż program wykona operację przebudowania do końca. 
Aby można było śledzić w czasie rzeczywistym postęp prac wyświetlona zostanie aktualizowana na bieżąco strona informacyjna:

\myimage{0.4}{status.png}{Bieżący stan operacji przebudowywania bazy danych}

Gdy wszystkie operacje zostaną wykonane, użytkownik zostanie automatycznie przekierowany do strony głównej (<Rysunek 1>).

\newpage

\section{Praca z programem}
Nim wybierzemy <przeglądarka ruchów> należy skonfigurować program i utworzyć (ew. przebudować) bazę danych.

\subsection{Przeglądarka ruchów}

Przeglądarka ruchów służy do analizowania danych statystycznych dotyczących ruchów 
wybieranych przez graczy na początku partii (\emph{podczas otwarcia}).

Ogólne informacje (dane statystyczne) są wyświetlane na głównym ekranie:

\myimage{0.47}{moveBrowserMain.png}{Przeglądarka ruchów -- stan początkowy}

Odnośniki w nagłówku strony służą do nawigacji do strony głównej oraz do ekranu widocznego na <rysunku 5>. 
Dodatkowo istnieje możliwość przełączania między polską i angielską wersją językową (poprzez kliknięcie na odp. flagach).

Poniżej znajduje się informacja o łącznej liczbie gier o których program posiada informacje. Następne dwa wiersze pokazują
prawdopodobieństwo wygranej bieżącego gracza (w pokazanej sytuacji) oraz liczbę wystąpień bieżącej pozycji w bazie danych.

Odnośnik <Lista pasujących gier> prowadzi do ekranu opisanego w następnym podrozdziale <Lista gier>, gdzie wypisane będą
przykładowe gry z bazy danych, w których \underline{wystąpiła pozycja bieżąca}.

Po lewej głównej części strony znajduje się plansza ilustrująca bieżącą pozycje (na <rysunku 5> jest to sytuacja początkowa, 
na <rysunku 6> zostały wykonane 3 zagrania). Znakami 'x' oznaczone zostały ruchy, które zostały wykonane w chociaż jednej z
gier z kolekcji analizowanej przez program. 

Pozostały obszar zajmują dwie tabele, w których wyliczono ww. zagrania oraz podano informacje, które mają pozwolić na ocenę
ruchu (np. czy daje on dużą szansę wygranej).

Kliknięcie na dowolny 'x' na planszy bądź na algebraiczne oznaczenie ruchu (kolumna \emph{Ruch}) przenosi użytkownika do
analizy sytuacji, w których to właśnie ten ruch został zagrany (por. <rysunek 6> i jego opis).

Kliknięcie na odnośnik z kolumny \emph{Łącznie gier} prowadzi do ekranu opisanego w następnym podrozdziale <Lista gier>, 
gdzie wypisane będą przykładowe gry z bazy danych, w których \\ 
\underline{wystąpiła pozycja powstająca po zagraniu ruchu, którego dotyczy dany wiersz tabeli}.

\myimage{0.47}{moveBrowserMoveOver.png}{Przeglądarka ruchów -- podświetlone wiersze}

<Rysunek 6> pokazuje sytuację, w której zagrano już 3 ruchy. Gdy zagrany jest choć jeden ruch, pokazywane są dwa dodatkowe odnośniki:
\emph{Cofnij ostatni ruch} oraz \emph{Od nowa}. Pierwszy z nich cofa przebieg partii o jedno zagranie, drugi wraca na sam początek (pusta plansza).

Jeśli użytkownik ustawi kursor nad jednym ze znaków 'x', wówczas podświetlane są dane dotyczącego ruchu, któremu dany 'x' odpowiada.
Na <rysunku 6> widzimy sytuację, która powstałaby, gdyby użytkownik ustawił kursor w obszarze oznaczonym przez czerwoną gwiazdkę.

\newpage

\subsection{Lista gier}
Lista gier pozwala znaleźć gry w których pojawiła się wybrała pozycja. Udostępnianie dane o grach to nazwy graczy, ich rankingi oraz wynik partii.
Dodatkowo, kliknięcie odnośnika z kolumny \emph{odnośnik} prowadzi do strony <Analiza gry> opisanej w następnym podrozdziale. Odnośnik 
\emph{Pokaż bieżącą pozycję} prowadzi do <Przeglądarki ruchów> dla bieżącej pozycji.

\myimage{0.47}{gameList.png}{Lista gier}

\subsection{Analiza gry}
Analiza gry to podstrona na której można znaleźć szczegołowe informacje na temat jednej wybranej partii.

Dostępne opcje to:
1) informacje o zawodnikach, wyniku oraz dacie rozegrania partii
2) odnośnik do źródłowego pliku .sgf
3) ilustracja pozycji końcowej w danej grze
4) osadzona w stronę przeglądarka gier \emph{eidogo}

\myimage{0.47}{gameDetailsMain.png}{Analiza gry -- sytuacja początkowa}

\subsubsection{Przeglądarka eidogo}
  Eidogo to przeglądarka zapisów w formacie .sgf, dzięki której można w wygodny sposób zapoznać się z przebiegiem całej partii. 
  Na <rysunku 8> widzimy sytuację bezpośrednio po wczytaniu strony, na <rysunku 9> zaś czerwonymi strzałkami oznaczono przyciski,
  której pozwalają (odpowiednio, od lewej do prawej) wrócić na początek gry, cofnąć jeden ruch, zobaczyć następny ruch,
  przejść do końcowej sytuacji.
   
\myimage{0.47}{gameDetails.png}{Analiza gry -- sytuacja po kilku ruchach}

% \section{Rozwiązania przykładowych problemów}

\end{document}
