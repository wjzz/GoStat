\documentclass[11pt,leqno]{article}
\usepackage[polish]{babel}
\usepackage[utf8]{inputenc}
%\usepackage{polski}
\usepackage[T1]{fontenc}
\frenchspacing
\usepackage{indentfirst}


\title{\LARGE Dokumentacja projektu \textbf{System Zapisów}\\
							Dokument Wizji}
\author{Marek Brzóska, Wojciech Jedynak, Krzysztof Sakwerda}
\date{Wrocław, \today}

\begin{document}

\maketitle 

\thispagestyle{empty}
\tableofcontents

\newpage
\thispagestyle{empty}
\begin{center}
\textbf{Historia zmian niniejszego dokumentu}
\renewcommand{\arraystretch}{1.5}
\begin{tabular}{| c || c | c | c | } \hline
Data & Wersja & Autor & Opis zmian \\ \hline
2009/10/20 & 1.0 & Zespół & Utworzenie Dokumentu \\ \hline
2009/10/23 & 1.1 & Wojciech Jedynak & Ogólny opis produktu \\ \hline
2009/10/24 & 1.2 & Wojciech Jedynak & Cechy produktu \\ \hline
2009/10/25 & 1.3 & Marek Brzóska & Podstawowe przypadki użycia \\ \hline
2009/10/26 & 1.4 & Marek Brzóska & Inne wymagania produktu \\ \hline
2009/10/31 & 1.5 & Krzysztof Sakwerda & Wstęp \\ \hline
2009/11/06 & 1.6 & Krzysztof Sakwerda & Opis użytkownika \\ \hline
2009/11/07 & 1.7 & Krzysztof Sakwerda & Korekta, skład w sytemie \LaTeX \\ \hline
\end{tabular}
\end{center}

\newpage
\section{Wprowadzenie}

\subsection{Cel dokumentu wizji}
Celem niniejszego dokumentu jest ogólny opis wymagań stawianych aplikacji, ze względu na jej przeznaczenie i sposób użycia oraz określenie najważniejszych założeń realizacji projektu. Nie są tu podejmowane żadne decyzje implementacyjne, zostaną one opisane w osobnych dokumentach.

\subsection{Ogólny opis produktu}
Produkt ma na celu umożliwienie studentom zdalne zapisywanie się na zajęcia tj. wybór przedmiotów oraz grup ćwiczeniowych, do których student chce uczęsczać. Podczas trwania zapisów student nie będzie musiał cały czas siedzieć przy komputerze, będzie miał możliwość wybrania kilu interesujących go planów i zostanie poinformowany o przebiegu zapisów. Jeśli nie uda mu się zapisać na wszystkie wybrane zajęcia, bądź w ogóle ułożyć planu będzie mógł podjąć stosowne kroki w kolejnej iteracji systemu zapisów. Ponadto chcemy ułatwić studentom wybór przedmiotu oraz zweryfikowanie wymagań stawianych przez prowadzących poprzez 
\\ \\
Głównym założeniem produktu jest umożliwienie studentom całkowicie zdalnych zapisów na zającie bez konieczności fizycznej obecności przy komputerze podczas ich trwania.

\section{Opis użytkownika}

\subsection{Dane statystyczne dotyczące użytkowników i rynku}
Z przeprowadzonych przez naś wśród studentów badań wynika, że praca z obecnie działającym systemem zapisów jest odrobinę uciążliwa. Trzy terminy otwarcia (odległe od siebie o 24 godziny) wymuszają konieczność obecności przy komputerze o określonej godzinie przez trzy dni z rzędu. Nie jest to zbyt wygodne i nie zawsze możliwe. Część studentów rozwiązue ten problem pisząc programy, które za nich będą zapisywać się na zajęcia. Jednak napisanie takiego programu to koszt czasowy, który student musi ponieść, oraz dodatkowe obciążenie dla systemu zapisów (przy zapisach na semest letni w roku akademickim 2008/2009 były z tego powodu problemy). \\
Zdecydowana większość (około 75 - 80 \%) ankietowanych stwierdziła, że proponowane przez nas rozwiązanie jest korzystne. \\
Spora część (około 60 \%) badanych uznała również, że interesujące  i pomocne byłoby dla nich umożliwienie przeglądu oceny zajęc poprzez system zapisów (a nie jak jest obecnie poprzez strony internetowe prowadzących) oraz możliwość weryfikacji wymagań byłaby pomocna podczas zapisów na zajęcia.

\subsection{Opis użytkowników}
W systemie zapisów rozróżniane będą trzy kategorie użytkowników:
\begin{itemize}
\item Użytkownik - administrator \\
Jego zadaniem ma być umieszczanie oferty dydaktycznej, kontrolowanie kolejnych faz pracy systemu oraz umożliwienie studentom głosowania na ofertę dydaktyczną oraz oceny zajęć. \\
\item Użytkownik - pracownik \\
Ma możliwość umieszczania w systemie propozycji przedmiotów, obejrzenia wyników głosowania oraz kontaktu mailowego ze studentami zapisanymi do jego grupy.
\item Użytkownik - student \\
Będzie miał możliwość przeglądu oferty dydaktycznej wraz z oceną zajęć, zapisów na zajęcia oraz przeglądania już utworzonych grup, a także planów zajęć innych studentów oraz pracowników uczelni.\\
\end{itemize}

\subsection{Środowisko użytkowników}
Użytkownikami systemu zapisów bedą studenci uczelni (wydziału), która zdecyduje się na wdrożenie naszego systemu.

\subsection{Podstawowe potrzeby użytkowników}
Jak już wspomniano student zapisując się na zajęcia chciałby móc ułożyć swój plan zajęć (być może w kilku wersjach) jednokrotnie w ciągu pewnego okresu czasu i pozostawienie właściwych zapisów zatomatyzowanemu systemowi. Ponadto chciałby mieć możwiwość zarówno wglądu jak i wyrażenia opinii o zajęciach, na które chciałby uczęszczać, bądź uczęszczał. Przydałaby mu się również możliwość weryfikacji stawianych przez prowadzących wymagań (czasami wiedza z wymaganych przedmiotów jest wykorzystywana w tak niewielkim stopniu, że student poradzi sobie nabywając wiadomości we własnym zakresie). \\
Proponowane przez nas rozwiązanie ma wyjśc na przeciw powyższym oczekiwaniom i zwiększyć wygodę zapisów na zajęcia.

\subsection{Rozwiązania alternatywne i konkurencyjne}
Obecnie jest używany system zdalnych zapisów na zajęcia, nie zaspokaja on jednak potrzeb użytkownika, o których mowa w poprzednim punkcie. Przerowadzone badania wykazują, że nasze rozwiązanie w znaczący sposób poprawi komfort użytkownika systemu.

\section{Ogólny opis produktu}
Aplikacja ma na celu umożliwienie użytkownikowi wykonanie czynności związanych z układaniem planu zajęć. Dostęp do aplikacji będzie się odbywał poprzez Internet za pomocą przeglądarki internetowej. Użytkownik - pracownik będzie mógł wprowadzić do systemu propozycje przedmiotów, sprawdzić wyniki głosowania oraz skontaktować się ze studentami zapisanymi na jego przedmioty. Użytkownik-student będzie mógł głosować na przedmioty zaproponowane przez pracowników, zapisywać się na przedmioty wybrane do realizacji oraz oceniać przedmioty na które uczęszczał. Użytkownik - administrator będzie miał możliwość nadzorowania wszystkich etapów i będzie mógł wprowadzać ręczne poprawki.

\subsection{Schemat produktu}
Aplikacja będzie podzielona na współpracujące ze sobą moduły. Moduły będą ze sobą powiązanie z ten sposób, że wyniki uzyskane i zebrane przez dany moduł będą danymi wejściowymi dla następnego modułu.
Lista modułów:
\begin{itemize}
\item Propozycje przedmiotów
\item Głosowanie na przedmioty
\item Układadnie planów
\item Dopasowywanie planów (przez system)
\item System kolejkowania
\item Ocena zajęć
\end{itemize}

\subsection{Określenie pozycji produktu na rynku}
Serwis przeznaczony jest dla Studentów i Pracowników Instytutu Informatyki Uniwersytetu Wrocławskiego. Celem projektu jest stworzenie systemu wygodniejszego od systemu używanego do tej pory. Dodatkowo chcemy, aby z systemem zapisów zintegrowane były inne aplikacje (np. Ocena zajęć). Dzięki temu użytkownicy będą mieli dostęp do większej liczby informacji podczas podejmowania decyzji.

\subsection{Podsumowanie możliwości}
Opis formatu:
\begin{itemize}
\item Możliwość \\
\begin{itemize}
\item Lista cech, które ją realizują
\end{itemize}

\item Student nie musi być dostępny przy komputerze o (z jego punktu widzenia) losowej porze, może uzupełnić swoje preferencje o dowolnej porze w ustalonym okresie (trwającym około 2 tygodnie)
\begin{itemize}
\item Możliwość układania alternatywnych planów
\end{itemize}

\item Student nie musi odwiedzać wiele razy dziennie serwisu, żeby sprawdzić czy zwolniły się miejsca w grupie, która go interesuje
\begin{itemize}
\item Kolejkowanie
\end{itemize}

\item Student może sprawdzić jaką wiedzą powienien dysponować przed zapisaniem się na dany przedmiot oraz na jakich przedmiotach zdobywa się tę wiedzę.
\begin{itemize}
\item Propozycje przedmiotów (perspektywa wykładowcy), głosowanie (perspektywa studentów)
\end{itemize}

\item Dyrektor do spraw dydaktycznych ma wgląd do danych liczbowych takich jak liczba studentów oczekujących na zwolnienie się miejsca z danego przedmiotu i może na to odpowiednio reagować (np. otwierająć nowe grupy, zwiększająć limity).
\begin{itemize}
\item Panel sterowania dla administatora systemu
\end{itemize}
\end{itemize}

\subsection{Założenia i zależności}
Zakładamy, że w danym roku akademickim z aplikacji będzie korzystać około 600 studentów.

\subsection{Koszty i ceny}
TODO

\section{Cechy produktu}

\subsection{Zgłaszanie propozycji przedmiotów, które mają być poddane głosowaniu}
Pracownicy będą mogli zgłaszać propozycje przedmiotów, które chcieliby poprowadzić w przyszłym roku akademickim. Do informacji podawanych podczas zgłoszenia należą: nazwa przedmiotu, typ przedmiotu, poziom zaawansowania, wymagania wstępne, wstępny program zajęć, literatura.

\subsection{Głosowanie na przedmioty}
Studenci będą mogli przeczytać informacje o przedmiotach zaproponowanych przez pracowników i następnie będą mogli wyrazić swoje zainteresowanie danym przedmiotem. Dzięki temu Dyrektor dydaktyczny będzie widział, które przedmioty cieszą się dużym zainteresowaniem (i znajdzie prowadzących dla wielu grup); które warto zorganizować, bo uzbiera się pełna grupa; które będą świecić pustkami.

\subsection{Edytor planów}
Studenci będą mieli do dyspozycji graficzny edytor planów, którym pozwoli im w wygodny sposób wypróbować wiele możliwości. Przykładowo: w edytorze widoczne będzie, że zajęcia w ramach kilku przedmiotów na siebie zachodzą.

\subsection{Dopasowywanie planów}
Gdy minie termin na tworzenie planów indywidualnych, uruchomiony będzie modul dopasowywujący plany. Jego zadaniem będzie przeglądać plany stworzone przez studentów (w kolejności malejących priorytetów) i zachłanne znajdowanie planu, który jest realizowalny (są wolne miejsca).

\subsection{Kolejkowanie}
TODO

\subsection{Ocena zajęć}
TODO

\section{Atrybuty cech}
Opis atrybutów: \\
\textbf{Status} - Ustalony po negocjacjach i przeglądzie dokonanym przez zespół programistów. Informacje statusu umożliwiają śledzenie postepu podczas definiowania linii bazowej przedsiewziecia.\\
\textbf{Priorytet} - Priorytet wprowadzenia do projektu danej cechy.\\
\textbf{Ryzyko} - Ustalane przez zespół programistów, na podstawie pradopodobieństwa, że w przedsięwzięciu wystąpią niepożądane zdarzenia typu przekroczenie kosztów,opóznienie planu lub anulowanie. \\
\textbf{Stabilność} - Ustalana przez analityka i zespół programistów, na podstawie prawdopodobieństwa, że zmieni się cecha lub zmieni się rozumienie cechy przez czlonków zespołu programistów. Ta informacja jest stosowana, aby pomóc ustalic priorytety tworzenia oprogramowania oraz określić te pozycje, dla których dodatkowe uzyskanie wymagań jest nastepnym własciwym zadaniem. \\
\textbf{Wersja docelowa} - Rejestrowana jest planowana wersja produktu, w której cecha pojawi się po raz pierwszy. To pole może byc użyte do przydzielenia cech poszczególnej wersji linii bazowej. Kiedy wersja docelowa jest połączona z polem statusu, zespól może proponować, zapisywać i omawiać różne cechy wersji bez zobowiazania do ich zapewnienia. Implementowane będą tylko te cechy, których status jest ustalony jako \"wprowadzona\" i dla których zdefiniowana jest wersja docelowa. \\ \\
\textbf{Zestawienie atrybutów cech} 
\subsection{Zgłaszanie propozycji przedmiotów, które mają być poddane głosowaniu}
Status: Proponowane  \\
Priorytet: Przydatne \\
Ryzyko: Niskie \\
Stabilność: Wysoka \\
Wersja docelowa: beta \\
\subsection{Głosowanie na przedmioty}
Status: Proponowane \\
Priorytet: Przydatne \\
Ryzyko: Niskie \\
Stabilność: Średnia \\
Wersja docelowa: beta \\
\subsection{Edytor planów}
Status: Zatwierdzone \\
Priorytet: Konieczne \\
Ryzyko: Duże \\
Stabilność: Niska \\
Wersja docelowa: alfa \\
\subsection{Dopasowywanie planów}
Status: Zatwierdzone \\
Priorytet: Konieczne \\
Ryzyko: Średnie \\
Stabilność: Niska \\
Wersja docelowa: prototyp \\
\subsection{Kolejkowanie}
Status: Zatwierdzone \\
Priorytet: Ważne \\
Ryzyko: Średnie \\
Stabilność: Niska \\
Wersja docelowa: prototyp \\
\subsection{Ocena zajęć}
Status: Proponowane \\
Priorytet: Przydatne \\
Ryzyko: Niskie \\
Stabilność: Pewna \\
Wersja docelowa: beta 

\section{Podstawowe przypadki użycia}

\begin{itemize}
\item Kształtowanie oferty
\begin{itemize}
\item Podawanie przedmiotów pod głosowanie \\
Użytkownik-pracownik może dodawać przedmioty. Każdy przedmiot powinien mieć opis, wymagania wstępne, liczbę punktów ECTS itp.

\item Głosowanie na przedmioty \\
Użytkownik-student może oddawać głosy na przedmioty zaproponowane przez użytkowników-pracowników. Liczba przedmiotów na które można zagłosować jest ograniczona. \\ \\

Podawanie przedmiotów pod głosowanie powinno zasadniczo być zakończone.
\end{itemize}

\item Układanie indywidualnego planu zajęć \\
Użytkownik-student układa zero lub więcej planów. System będzie próbował wybrać plan o najwyższym priorytecie. Do każdego planu można dodać listę grup do których system będzie zapisywał opcjonalnie.

\item Korygowanie indywidualnego planu zajęć \\
Wypisywanie się z zajęć, zapisywanie się do list oczekujący na zajęcia.

\item Ocena zajęć \\
Ocena zajęć, na które użytkownik-student był zapisany w poprzednim semestrze.
\end{itemize}

\section{Inne wymagania produktu}

\subsection{Spełniane normy}
Produkt ma spełniać normy organizacji W3C, to jest kod strony zgodny z XHTML1.0, arkusze stylów zgodne z CSS 2.0. Interfejs graficzny ma być intuicyjnie, względnie podobny do istniejącego w chwili obecnej Systemu Zapisów na Instytucie Informatyki Uniwersytetu Wrocławskiego.

\subsection{Wymagania stawiane systemowi}

System ma być w stanie obsłużyć jednocześnie 1000 studentów w wersji podstawowej. Interfejs musi być wystarczająco intuicyjny, żeby obyty z internetem użytkownik - student był w stanie opanować jego obsługę w ciągu kilku minut.

\subsection{Licencjonowanie i instalacja}
System będzie wymagał niezależnego zainstalowania serwera baz danych, serwera WWW, interpretera języka Python wraz ze ściśle określonym zestawem bibliotek. Instalacja produktu będzie polegała utworzeniu odpowiedniego katalogu, rozpakowaniu w tym katalogu archiwum z plikami programu oraz edycji plików konfiguracyjnych w celu przekazania systemowi danych niezbędnych do pracy z zewnętrznymi aplikacjami. \\ \\
System rozpowszechniany wraz z kodem źródłowym. Licencja będzie udzielana dożywotnio, na jedną maszynę, z możliwością dowolnej modyfikacji kodu, bez możliwości odsprzedaży czy przekazania żadnej jego części.

\subsection{Wymagania efektywnościowe}
Szybkość działania powinna pozostać na poziomie akceptowanym przez przeciętnego użytkownika, tj. odpowiedź serwera na dowolne zapytanie użytkownika nie powinna trwać więcej niż sekundę. Na większość zapytań system powinien odpowiadać \"od razu\".

\section{Wymagania dokumentacyjne}

\subsection{Wymagania dokumentacyjne}
Dokumentacja w formie podręczników dla użytkownika. Dodatkowo czytelnie sformatowany i udokumentowany zgodnie z konwencją kod w języku Python.

\subsection{Pomoc on line}

Nie przewidziana. Ewentualnie w formie osobnej usługi.

\subsection{Porady dotyczące instalacji, konfiguracji oraz plików informacyjnych}

TODO

\section{Słownik}

TODO

\end{document}
