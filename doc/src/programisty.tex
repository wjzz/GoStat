\documentclass[10pt,leqno]{article}
\usepackage[polish]{babel}
\usepackage[utf8]{inputenc}
\usepackage[T1]{fontenc}
\frenchspacing
%\usepackage{indentfirst}
\usepackage{listings}
\usepackage{hyperref}
\usepackage{framed}
\usepackage[textwidth=14cm,textheight=24cm]{geometry}

\usepackage{fancyhdr}
\pagestyle{fancy}
%\fancyhf{}
\cfoot{\thepage}
\lhead{\nouppercase{\leftmark}}
\rhead{\nouppercase{\rightmark}}

\newcommand{\cmd}[1]{
  \texttt{#1}
}

\title{\LARGE Licencjacki projekt programistyczny 2011 \\ 
       \ \\
       Program do wykonywania obliczeń statystycznych związanych z grą go \\ 
       \ \\
       Dokumentacja programisty }

% TODO autor wieksza czcionka

\author{Wojciech Jedynak}
\date{Wrocław, \today}

\begin{document}

\maketitle 

\newpage

\tableofcontents

\newpage

\section{Wprowadzenie}

\subsection{Cel dokumentacji}
Celem niniejszego dokumentu jest takie przedstawienie struktury projektu GoStat, aby umożliwić jego modyfikacje oraz utrzymywanie programistom, którzy
znają Haskella, ale nie należeli do początkowego zespołu. Wykaz użytych bibliotek powinien był pomocny, jeśli instalacja oprogrogramowania nie powiedzie 
się i konieczna będzie kompilacja programu ze źródeł. Dodatkowo życzeniem autora jest nakreślenie wykonanej pracy tak, aby zainteresowane osoby
były w stanie (w razie potrzeby) na wykorzystanie opisanych tu rozwiązań w swoich projektach.

\section{Organizacja projektu}

\subsection{Ogólny opis}
Program został napisany niemal w całości w Haskellu \cite{haskell}. 
Po uruchomieniu pliku \cmd{GoStat} wewnętrzny serwer HTTP nasłuchuje na porcie 8000, 
a komunikacja z użytkownikiem odbywa się za pomocą interfejsu WWW. 
Lista katalogów, w których znajduje się kolekcja plików SGF, zapisywana jest do pliku konfiguracyjnego,
wstępnie przetworzone (\emph{znormalizowane}) gry są przechowywane w bazie danych. Dialog z użytkownikiem może odbywać się w języku
polskim bądź angielskim.

\subsection{Konfiguracja programu}

W programie potrzebujemy przechowywać dwie informacje: jakiej bazy danych używa (chce używać) użytkownik i gdzie znajduje się jego
kolekcja zapisów partii go, które chciałby analizować naszym programem. Do przechowywania ww. danych używamy bardzo prostego, 
autorskiego formatu. Funkcje związane z wczytywanien, analizą leksykalną oraz zapisywaniem znajdują się w module \emph{Configuration}
(w pliku \cmd{src/Configuration.hs}).

\subsubsection{Opis formatu pliku CONFIG}

Format pliku jest opisywany przez poniższą gramatykę w postaci EBNF:
\begin{framed}
\noindent config ::= declaration* \\ 
declaration ::= db | dirs | '--' *anything* \\
db = dbserver ':' dbversion \\ 
dbversion = sqlite3 ';' path ';' | postgresql ';' \\
dirs = gamedirs ':' path ';'
\end{framed}

Przykładowy plik konfiguracyjny:

\begin{framed}
\noindent \--\-- new config \\
\--\--dbserver:postgresql; \\ 
dbserver:sqlite3;/home/wojtek/db/games.db; \\
gamedirs:/home/wojtek/data/;
\end{framed}

%\begin{minipage}[pos]{width} text 
%\end{minipage}



\subsection{Baza danych}

% Program pozwala na użycie baz Sqlite3 oraz PostgreSQL. 
% Możliwa jest zmiana decyzji co do tego, która z nich jest używana; 
% wymagane jest wówczas przebudowanie zawartości, 
% gdyż protokół komunikacyjny bazy PostgreSQL nie jest kompatybilny z protokołem
% bazy Sqlite3. 

Program używa bazy danych Sqlite3 \cite{sqlite3}. Programista nie musi zajmować
 się ręczną administracją bazy danych: służy do tego moduł \emph{DB} w pliku \cmd{src/DB.hs}. 

Używana jest jedna tabela o nazwie \cmd{go\_stat\_data}.

\subsubsection{Tabela go\_stat\_data}

\begin{center}
\textbf{Opis pól tabeli go\_stat\_data}
\renewcommand{\arraystretch}{1.5}
\begin{tabular}{| c | c | c | c | } \hline
 Pole    & Typ          & NULL dozwolone? & Opis                  \\ \hline
 id      & PRIMARY KEY  & nie & unikatowy identyfikator gry       \\ \hline
 winner  & CHAR         & nie & zwycięzca gry ('b' lub 'w')       \\ \hline
 moves   & VARCHAR(700) & nie & znormalizowany przebieg rozgrywki \\ \hline
 path    & VARCHAR(255) & nie & bezwględna scieżka do gry         \\ \hline
 b\_name & VARCHAR(30)  & nie & pseudonim (nazwisko) czarnego     \\ \hline
 w\_name & VARCHAR(30)  & nie & pseudonim (nazwisko) białego      \\ \hline
 b\_rank & VARCHAR(10)  & tak & ranking czarnego                  \\ \hline
 w\_rank & VARCHAR(10)  & tak & ranking białego                   \\ \hline
\end{tabular}

\end{center}

Jedyne pola, która nie są wymagane to pola b\_rank i w\_rank. Wynika to z tego, że na niektórych serwerach do gry w go
nie jest wymagane podanie swojego orientacyjnego poziomu ani rankingu. 

\subsection{Struktura modułów}
Lista modułów wchodzących w skład projektu:

\subsubsection{Data.SGF.Types i Data.SGF.Parsing}

\begin{framed}
\noindent type Move  = (Int, Int) \\
type Moves = [Move] \\
type PlayerName = String \\ 
data Winner     = Black | White \\
data Result = Unfinished | Draw | Win Winner PlayerName \\

\noindent parseSGF :: String -> Either String SGF \\
getResult :: SGF -> Result \\
isWithHandicap :: SGF -> Bool \\
getBlack, getWhite, getBlackRank, getWhiteRank, date :: SGF -> String
\end{framed}

Moduły \emph{Data.Sgf.Types} i \emph{Data.SGF.Parsing} zawierają deklaracje typów służących do 
przechowywania informacji o danej partii (m.in. wynik, dane graczy, lista ruchów) oraz funkcje, które
pozwalają dane te wyświetlać i analizować. 

\subsubsection{Analiza leksykalna (Parsec)}

\subsubsection{Transformations}
Przekształcenia matematyczne i normalizacja ruchów.

\begin{framed}
\noindent normalizeMoves :: [Move] -> [Move] \\

\noindent isOnMainDiagonal :: Move -> Bool  \\
isAboveMainDiagonal :: Move -> Bool \\
isBelowMainDiagonal :: Move -> Bool \\
isOnHorizontal :: Move -> Bool \\
isAboveHorizontal :: Move -> Bool \\
isBelowHorizontal :: Move -> Bool

\noindent horizontal :: Move -> Move \\
rotate90 :: Move -> Move \\
mainDiagonalMirror :: Move -> Move \\

\noindent transformIntoFirst :: Triangle -> (Move -> Move) \\
getTransformation :: Move -> (Move -> Move) \\

\noindent triangles :: [Triangle] \\
findTriangles :: Move -> [Triangle] \\
findTriangle :: Move -> Triangle
\end{framed}

\subsubsection{SgfBatching}
Konwersja plików SGF do formatu, który będzie łatwo zapisać w bazie danych.

\subsubsection{Lang}
Komunikaty w języku polskim i angielskim.

\subsubsection{Configuration}
Obsługa pliku CONFIG.

\subsubsection{Pages}
Tworzenie szablonów stron WWW.
\subsubsection{Generowanie html (xhtml 3000)}
\cite{xhtml}

\cite{jquery}
\cite{jqueryui}

\subsubsection{DB}
Zarządzanie bazą danych.

\subsubsection{Zarządzanie bazą danych (HDBC, HDBC-sqlite3)}

\begin{framed}
\noindent createDB :: GoStatM () \\
deleteDB :: GoStatM () \\
addFilesToDB :: GoStatM () \\
queryCountDB :: GoStatM Int \\
queryStatsDB :: String -> GoStatM [(String, Int, Int, Int)] \\
queryCurrStatsDB :: String -> GoStatM (Int, Int, Int) \\
queryGamesListDB :: String -> Int -> GoStatM [(Int, FilePath, String, String, String, String, String)] \\
queryFindGameById :: Int -> GoStatM (Maybe (String, FilePath))
\end{framed}          

\subsubsection{Server}
Serwer HTTP. 

Serwer HTTP (Happstack)
\cite{happstack}

\subsubsection{Main}
Punkt startowy aplikacji.

\subsection{Pozostałe pliki}
Pliki css, javascript, obrazki

\section{Kompilacja}

W głownym katalogu znajduje się plik \cmd{GoStat.cabal}. Pozwala on na wykorzystanie do kompilacji 
narzędzia Cabal \cite{cabal}, dzięki czemu:

\begin{enumerate}
\item Nie musimy sami dbać o to, aby zainstalowane zostały odpowiednie wersje pakietów z serwisu
  Hackage \cite{hackage}.
\item Cały projekt możemy skonfigurować i skompilować jednym poleceniem (\cmd{cabal configure \&\& cabal build}).
\item Program można zainstalować jednym poleceniem (\cmd{cabal install}).
\item Jednym poleceniem możemy utworzyć archiwum zawierające wszystkie pliki wykorzystywane w projekcie (\cmd{cabal sdist}).
\item Wszystkie pliki tworzone podczas komplilacji (*.o, *.hi) są umieszczane w katalogu \cmd{dist} -- pozwala
  to na utrzymanie ładu w strukturze katalogów należących do projektu.
\end{enumerate}

\noindent Dodatkowo, aby zautomatyzować i uprościć pewne często wykonywane czynności, utworzono plik \cmd{Makefile}, który
pozwala na wydanie następujących poleceń:

\begin{itemize}

\item \cmd{make} -- Kompilacja projektu, zbudowanie programu wynikowego
\item \cmd{make run} -- Uruchomienie programu
\item \cmd{make test} -- Wykonanie wszystkich testów
\item \cmd{make install} -- Instalacja programu ze źródeł
\item \cmd{make windows-release} -- Utworzenie pliku \cmd{dist/GoStat-binary-windows.tar.gz}
\item \cmd{make linux-release}   -- Utworzenie pliku \cmd{dist/GoStat-binary-linux.tar.gz}
\end{itemize}

\noindent Archiwa utworzone poprzez \cmd{make \{windows, linux\}-release} zawierają dokumentację, plik wykonywalny \cmd{GoStat}
oraz wszystkie pliki dodatkowe, niezbędne do uruchomienia programu (pliki .css, .js, obrazki itp). \\

\noindent Dokumentacja tworzona jest przy pomocy programu \cmd{pdflatex}.

\section{Testy}

\subsubsection{Testowanie (HUnit, QuickCheck, test-framework)}

\section{Wykorzystane narzędzia pomocnicze}

W niniejszej sekcji wymieniono i pokrótce opisano najważniejsze narzędzia, które pozwoliły ukończyć projekt.

\subsection{Git i portal github}
Git \cite{git} to system do kontroli wersji autorstwa Linuxa Torwardsa (TODO: sprawdzić pisownię). 

Github \cite{github} to portal, który pozwala
na przechowywanie kodu źródłowego (ogólnie: repozytoriów kodu zarządzanych przez git) i udostępnienie go innym programistom.
Poprzez użycie tych zasobów rozwiązałem kwestię składowania projektu i mogłem swobodnie eksperymentować: nietrafione zmiany
można było wycofać jednym poleceniem.

\subsection{Latex}
System \LaTeX pozwolił na utworzenie dokumentacji w formacie pdf, który jest standardem w informatyce.

\begin{thebibliography}{9}

\bibitem{git}
  \emph{Git -- narzędzie do kontroli wersji} \\
  \url{http://git-scm.com/}

\bibitem{github}
  \emph{Portal github.com} \\
  \url{https://github.com/}

\bibitem{wjzz}
  \emph{Repozytorium projektu GoStat w portalu github} \\
  \url{https://github.com/wjzz}

\bibitem{sqlite3}
  \emph{SQLite3 -- lekka baza danych} \\
  \url{http://www.sqlite.org/}

\bibitem{jquery}
  \emph{jQuery -- biblioteka dla JavaScriptu} \\
  \url{http://jquery.com/}

\bibitem{jqueryui}
  \emph{jQuery IU -- biblioteka komponentów dla JavaScriptu} \\
  \url{http://jqueryui.com/}

\bibitem{eidogo}
  \emph{Eidogo -- interaktywna plansza} \\
  \url{http://eidogo.com/source}

\bibitem{haskell}
  \emph{Haskell -- strona główna} \\
  \url{http://www.haskell.org/}

\bibitem{ghc}
  \emph{Glasgow Haskell Compiler} \\
  \url{http://www.haskell.org/ghc/}

\bibitem{hackage}
  \emph{Hackage -- kolekcja pakietów Haskellowych} \\
  \url{http://hackage.haskell.org/}

\bibitem{cabal}
  \emph{Cabal -- narzędzie do tworzenia i instalowania pakietów Haskellowych} \\
  \url{http://www.haskell.org/cabal/}

\bibitem{happstack}
  \emph{Happstack -- serwer HTTP dla Haskella} \\
  \url{http://happstack.com/index.html}

\bibitem{xhtml}
  \emph{Biblioteka xhtml dla Haskella} \\
  \url{http://hackage.haskell.org/package/xhtml-3000.2.0.1}

\end{thebibliography}


\end{document}
