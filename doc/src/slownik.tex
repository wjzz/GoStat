\documentclass[12pt,leqno]{article}
\usepackage[polish]{babel}
\usepackage[utf8]{inputenc}
\usepackage[T1]{fontenc}
\frenchspacing
%\usepackage{indentfirst}
\usepackage{listings}
\usepackage{framed}
\usepackage[textwidth=14cm,textheight=24cm]{geometry}
\usepackage{hanging}

\title{\LARGE Licencjacki projekt programistyczny 2011 \\ 
       \ \\
       Program do wykonywania obliczeń statystycznych związanych z grą go \\ 
       \ \\
       Słownik pojęć }

\newcommand{\entry}[2]{
 \begin{hangparas}{.25in}{1}
  \noindent \textbf{#1} -- #2 \\
 \end{hangparas}
}

\usepackage{fancyhdr}
\pagestyle{fancy}
%\fancyhf{}
\cfoot{\thepage}
\lhead{\nouppercase{\leftmark}}
\rhead{\nouppercase{\rightmark}}

% TODO autor wieksza czcionka

\author{Wojciech Jedynak}
\date{Wrocław, \today}

\begin{document}

\maketitle 

\thispagestyle{empty}

\newpage

\section{Słownik}

\subsection{Pojęcia goistyczne}

\entry{Go}{Gra planszowa dla dwóch osób, rozgrywana na kwadratowej planszy.}

\entry{Gracz}{Osoba rozgrywaja partię go}

\entry{Kamienie}{Sprzęt do gry w go; pojęcię używane zwyczajowo zamiast \emph{pionek}. Wyróżniamy kamienie
 białe i czarne. Gracz rozpoczynający rozgrywkę używa kamieni czarnych.}

\entry{Czarny}{Gracz, który rozpoczyna grę go i używa kamieni koloru czarnego}

\entry{Biały}{Przeciwnik gracza czarnego, używa kamieni koloru białego}

\entry{Zapis gry}{Przebieg partii go, zapisany w sposób symboliczny lub cyfrowy. Zazwyczaj zawiera
  dodatkowe informacje takie jak: wynik końcowy, data i miejsce rozgrywki, informacje o graczach, wariant zasad.}

\entry{SGF (ang. \emph{Smart Go Format})}{Tekstowy format do do przechowywania zapisów gier go.}

\entry{Serwer do gry w go}{Serwis internetowy, który umożliwia grę w go z innymi graczami (zazwyczaj z całego świata).}

\entry{Ranking (gracza), siła}{Przyznawany przez wyznaczoną do tego organizację (lub algorytm w przypadku serwerów do gry w go)
  stopień mający określić poziom umiejętności danego zawodnika. Zazwyczaj używa się skali kyu/dan: początkujący gracz
  otrzymuje stopień 30 kyu (w skrócie 30k). Lepsi od niego mają stopnie o \emph{niższym} numerze; gracz o rankingu (\emph{sile})
  5k powinien zazwyczaj wygrywać z graczem o sile 6k. Skala kyu kończy się na pierwszym kyu (1k), dalej przyznawane są stopnie
  w skali dan -- są to stopnie mistrzowskie. W skali dan sytuacja się odwraca: 1 dan (1d) to najniższy stopień w tej skali,
  najwyższy to 7 dan (w skali amatorskiej) bądź 9 dan (w skali zawodowców).}

\subsection{Terminy matematyczne}

\entry{Symetria}{}

\entry{Normalizacja (ruchu, gry)}{}

\subsection{Terminy informatyczne}

\entry{Interfejs WWW}{}

\entry{HTTP (ang. \emph{Hypertext Transfer Protocol})}{Protokół internetowy do przesyłania dokumentów hipertekstowych}

\entry{Baza danych}{}

\entry{Moduł}{}

\entry{Pakiet, pakiet Haskellowy}{}

\entry{Haskell}{}

\entry{Paradygmat funkcyjny}{}

\entry{Język funkcyjny}{}

\entry{Ścieżka bezwględna}{}

\entry{Ścieżka wględna}{}

gramatyka bezkontekstowa \\
postac EBNF \\ 

\subsection{Pojęcia ściśle związane z projektem}

\entry{GoStat}{}

\entry{Plik konfiguracyjny}{}

\end{document}
